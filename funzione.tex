
\section{Studio di funzione}	
	 	
	 	\subsection{Massimi e minimi di funzioni a una variabile}
	 	
	 	Un punto $x_0\in[a,b]\leq \text{Dominio \f}$ si dice:\\
	 	
		\paragraph{Massimo locale} esiste un intorno di \f(x) in cui si ha 
		
		\[
		\f(x)\geq\text{\f}(x_0)
		\]
		
		\paragraph{Minimo locale} esiste un intorno di \f(x) in cui si ha
		
		\[
		\f(x)\leq \text{\f}(x_0)
		\]
		
		Se è anche $\forall x \in \delta$ allora è anche 
		\textbf{Assoluto}
		
		\subsubsection{Ricerca dei massimi e minimi locali}
		
		\paragraph{in $\f(x)$ derivabile}
		
		Si cercano le soluzioni dell'equazione ${\f}'(x)=0$ per ogni soluzione $x_0$ si studia il segno di ${\f}'(x)$ per un intorno di $x_0$\\
		Se ${\f}'(x)>o$ a sinistra e ${\f}'(x)<0$ a destra è \textbf{massimo locale}\\
		Se ${\f}'(x)<0$ a sinistra e ${\f}'(x)>o$ a destra è \textbf{minimo locale}\\
		Se ${\f}'(x)$ è sempre o maggiore o minore di 0 allora può essere un \textit{flesso}
		
		\subsubsection{Formule di Taylor}
		
		Sia $\f(x)\in c^n(a,b)$ e $x_0 \in (a,b)$ chiamasi \textbf{polinomio di Taylor} $p_n(x)$ della funzione $\f(x)$ nel punto $x_0$ il polinomio
		\[
		\sum\limits_{i=0}^n {h \frac{\text{\f}^h(x_2)}{h!}(x-x_0)^h}
		\]
		dove si intende $\f^0(x)0\text{\f}(x)$ e 0!=1
		
		\subsection{Funzioni a più variabili}
		
\subsubsection{Piano tangente}
Data una funzione $\text{\f}(x,y)$ ed un punto $P_0(x_0,y_0)$

\[
z=\f(x_0,y_0)+{\f}_x(x_0,y_0)*(x-x_0)+{\f}_y(x_0,y_0)*(y-y_0)
\]

\subsubsection{Gradiente}
Il gradiente di una funzione a due variabili si scrive come:

\[
{\nabla\text{\f}(x,y)}=\left({
\begin{array}{c}
  {\text{\f}_x(x,y)} \\
  {\text{\f}_y(x,y)}\\
\end{array}
}\right)
\]

Che non è altro che la direzione di massima crescita della derivata.
Per il calcolo consiglio la forma

\[
\left|{{\nabla\text{\f}(x,y)}}\right|
=\sqrt[2]{
  {(\text{\f}_x(x,y))}^2
+
  {(\text{\f}_y(x,y))}^2
}
\]

In forma unitaria o versore si scrive 
\[
\left({
\begin{array}{c}
  {\frac{\text{\f}_x(x,y)}{\left|{\nabla\text{\f}(x,y)}\right|}}\\
  {\frac{\text{\f}_y(x,y)}{\left|{\nabla\text{\f}(x,y)}\right|}}\\
\end{array}
}\right)
\]

\subsection{Massini e minimi a due variabili}

\subsubsection{Massimi e minimi relativi}
Per trovare i massimi e minimi relativi devo studiare il determinante della matrice Hessiana.

\begin{center}
\begin{tabular}{l|l|l}
\toprule
{$det[h(x_0,y_0)]$} & {$f_{xx}(x,y)$} & Quindi? \\
\midrule
$>0$ & $>0$ & min \\
$>0$ & $<0$ & max \\
$<0$ & $>=<0$ & sella \\
$=0$ & $>=<0$ & indefinito\\
\bottomrule
\end{tabular}
\end{center}

\subsubsection{Massimi e minimi assoluti}
Data un dominio

\begin{itemize}
    \item{trovo i punti in cui le derivate vanno a 0 tenendo solo quelli interni al dominio}
    \item{calcolo massimi e minimi relativi all'interno del dominio e il valore della funzione}
\end{itemize}
Con questi dati puoi studiare la funzione per ricavare i massimi e minimi assoluti ovvero:	

\begin{itemize}
    \item{studia la funzione sui punti delimitati dal dominio}
    \item{studiare la funzione nei vertici del dominio}
\end{itemize}
