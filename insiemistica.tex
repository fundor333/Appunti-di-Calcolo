\section{Cenni di insiemistica}
\textbf{Le nozioni di insieme di elementi di un insieme e di appartenenza vanno assunte come nozioni primitive.}\\

\subsection{Notazioni}
	\begin{math}
	 a\in A
	\end{math}
 Significa che è un elemento dell'insieme A\\
	\begin{math}
	 a\notin A
	\end{math}
 Significa che non è un elemento dell'insieme A\\
 Un insieme può essere definito espicitamente elencandone tutti i suoi elementi (ovviamente solo se è un insieme finito, cioè formato da un numero finito di elementi) oppure dando una legge oggettiva che permetta di identificare in maniera certa tutti i suoi elementi.
 
 \paragraph{Esempi}
 	\begin{math}
 	 A=\left\{{1,3,7}\right\}
 	\end{math}
 Rappresenta l'insieme i cui elementi sono i numeri 1,3,7\\
\begin{math}
 	 A=\left\{{2,4,6,8,10}\right\}
 	\end{math}
 Rappresenta l'insieme i cui elementi sono i numeri 2,4,6,8,10. Tale insieme potrebbe essere individuato in altro modo in base alla\\
 seguente legge:\\
\textit{ A è l'insieme formato da i numeri pari positivi minori o uguali a 10}\\
 	\begin{math}
 	 A=\left\{{\forall n \in N| n\geq100}\right\}
 	\end{math} 
 Si legge:\\
 \textit{ A è l'insieme di tutti i numeri interi (come vedremo più avanti l'insieme dei numeri naturali si indica con $\mathcal{N}$})\\
 
\subsection{Insiemi}

\paragraph{Sottoinsiemi impropri o sottoinsiemi}
 L'insieme A si dice sottoinsieme di B (e si scrive $A\subseteq B$) se ogni elemento di A è anche elemento di B.
 
\paragraph{Sottoinsiemi Propri}
 A si dice sottoinsieme proprio di B (e si scrive $A \subset B$) se ogni elemento di A appartiene a B e inoltre vi è almeno un elemento di B che non appartiene ad A\\

 \subsection{Operazioni sugli insiemi}

 \paragraph{Unione}
 Dati due insiemi A e B si definisce la loro unione (che si scrive $C=A \cup B$) l'insieme C formato da tutti gli elementi di A e da tutti gli elenemti di B (copiati una sola volta anche se presenti sia in A che in B).

 \paragraph{Esempio}
  \[
   A=\left\{{2,-5,4}\right\} B=\left\{{0,3,-5}\right\}\Rightarrow A\cup B=\left\{{2,-5,4,0,3}\right\}
  \]
  In maniera ovvia le operazioni di unione e di intersezione possono estendersi ad un numero qualsiasi di insiemi.

 \paragraph{Complemento di A in B}
  Si indica con $B/A$ ed è definito come l'insieme di tutti gli elementi di B che non appartengono ad A.

  \paragraph{Esempio}
  Riferendosi agli esempi precedenti si ha:
  \begin{math}
  B/A=\left\{{0,3}\right\}
  A/B=\left\{{2,4}\right\}
  \end{math}

 \paragraph{Proprietà delle operazioni sugli insiemi}
 \begin{gather*}
  A\cup B=B\cup A\\
  A\cup(B\cup C)=(A\cup B)\cup C\\
  A\cap B=B \cap A\\
  A\cap (B\cap C)=(A\cap B)\cap C\\
  A\cap (B\cup C)=(A\cup B)\cap (A\cup C)\\
  A\cup (B\cap C)=(A\cap B)\cup (A\cap C)\\
  C/ A\cup B=(C/A)\cap (C/B)\\
  C/ A\cap B=(C/A)\cup (C/B)\\ 
 \end{gather*}
 Due insiemi si dicono uguali se contengono gli stessi elementi. In tal caso valgono contemporaneamente
 	$A \subseteq B$
 	e
 	$A \supseteq B$\\
L'insieme che non contiene alcun elemento si chiama insieme vuoto $\phi$
Due insiemisi dicono disgiunti se la loro intersezione è l'insieme vuoto, cioè se:
\[
A\cap B=\phi
\]
Valgono le seguenti proprietà:
\begin{gather*}
	A\cup \phi=A \quad
	A\cap \phi=\phi \quad
	A/ \phi=A
 \end{gather*}

\paragraph{Prodotto Cartesiano}
Dati due insiemi A e B chiamasi prodotto cartesiano (e si indica con AxB) l'insieme delle coppie ordinate (a,b) con $a\in A$ e $b\in B$

\paragraph{Insiemi numerici}

 \paragraph{I numeri naturali $\mathbb{N}$}
  L'insieme $\mathbb{N}$ indica l'insieme dei numeri naturali. È chiuso rispetto alla somma e al prodotto; questo vuole dire che il risultato di una somma o di un prodotto è sempre in $\mathbb{N}$.

 \paragraph{I numeri interi con segno $\mathbb{Z}$}
  L'insieme $\mathbb{Z}$ indica l'insieme dei numeri interi relativi con segno. È chiuso per le operazioni di somma, prodotto e sottrazione.

  \paragraph{Insieme dei numeri razionali $\mathbb{Q}$}
  L'insieme $\mathbb{Q}$ indica l'insieme dei numeri razionali. È chiso rispetto somma, prodotto, sottrazione e divisione diversa da 0. I suoi numeri possono essere rappresentati come coppie di numeri interi in cui il primo è il numeratore e il secondo è il denominatore~$(n,d)$. Due numeri razionali $\frac{a}{b}$ e $\frac{a'}{b'}$ sono equivalenti solo se $ab'=a'b$. I numeri razionali possono anche rappresentare decimali finiti o periodici quali 
  $\frac{2}{3}=0, \overset {\scriptscriptstyle-}{6} $ oppure
  $\frac{3}{4}=0,75 $

 \paragraph{Insieme dei numeri reali $\mathbb{R}$}
 L'insieme $\mathbb{R}$ indica l'insieme dei numeri reali. Può essere espresso attraverso il concetto di limite o con il concetto di elemento di separazione di due classi contigue di numeri razionali.\\
 Per esempio $\sqrt[2]{2}$ può essere l'elemento di separazione fra la classe formata da tutti i numeri negativi, lo zero e tutti i numeri positiviil cui quadrato è pari o superiore a 2 e la classe formata da tutti i numeri il cui quadrato è superiore a 2.In oltre gode della proprietà di completezza ovvero che:
 \[
 a \leq \delta \leq b \quad \forall a \in A \quad \wedge \forall b \in B
 \]
 
